\documentclass[a4paper,10pt]{article}
\usepackage{eumat}

\begin{document}
\begin{eulernotebook}
\begin{eulercomment}
Nama  : Latifah Nurfitriyani\\
Kelas : Matematika E\\
NIM   : 22305141010\\
\end{eulercomment}
\eulersubheading{}
\begin{eulercomment}
\begin{eulercomment}
\eulerheading{Menggambar Plot 3D dengan EMT}
\begin{eulercomment}
Ini adalah pengenalan plot 3D di Euler. Kita perlu plot 3D untuk
memvisualisasikan fungsi dari dua variabel.

Euler menggambar fungsi tersebut menggunakan algoritma pengurutan
untuk menyembunyikan bagian-bagian di latar belakang. Secara umum,
Euler menggunakan proyeksi pusat. Defaultnya adalah dari kuadran
positif x-y menuju asal x=y=z=0, tetapi sudut=0° terlihat dari arah
sumbu y. Sudut pandang dan tinggi dapat diubah.

Euler dapat membuat plot

- permukaan dengan bayangan dan garis tingkat atau rentang tingkat,\\
- clouds of points,\\
- kurva parametrik,\\
- permukaan implisit.

Plot 3D dari sebuah fungsi menggunakan plot3d. Cara termudah adalah
merencanakan sebuah ekspresi dalam x dan y. Parameter r mengatur
jangkauan plot sekitar (0,0).
\end{eulercomment}
\begin{eulerprompt}
> aspect(1.5); plot3d("x^2+sin(y)",-5,5,0,6*pi):
\end{eulerprompt}
\eulerimg{17}{images/EMT4Plot3D_Latifah Nurfitriyani_22305141010-001.png}
\begin{eulerprompt}
> plot3d("x^2+x*sin(y)",-5,5,0,6*pi):
\end{eulerprompt}
\eulerimg{17}{images/EMT4Plot3D_Latifah Nurfitriyani_22305141010-002.png}
\begin{eulercomment}
Silakan lakukan modifikasi agar gambar "talang bergelombang" tersebut tidak lurus melainkan melengkung/melingkar, baik
melingkar secara mendatar maupun melingkar turun/naik (seperti papan peluncur pada kolam renang. Temukan rumusnya.
\end{eulercomment}
\eulerheading{Fungsi dari dua Variabel}
\begin{eulercomment}
Untuk grafik fungsi, gunakan

- ekspresi sederhana dalam x dan y,\\
- nama dari sebuah fungsi dari dua variabel\\
- atau matriks data.

Standarnya adalah kisi kawat yang diisi dengan warna yang berbeda di
kedua sisi. Perhatikan bahwa jumlah default interval grid adalah 10,
tetapi plot menggunakan jumlah default 40x40 persegi panjang untuk
membangun permukaan. Hal ini dapat diubah.

- n=40, n=[40,40]: jumlah garis kisi dalam setiap arah\\
- grid=10, grid=[10,10]: jumlah garis grid di setiap arah.

Kami menggunakan default n=40 dan grid=10.
\end{eulercomment}
\begin{eulerprompt}
> plot3d("x^2+y^2"):
\end{eulerprompt}
\eulerimg{17}{images/EMT4Plot3D_Latifah Nurfitriyani_22305141010-003.png}
\begin{eulercomment}
Interaksi pengguna dimungkinkan dengan \textgreater{}user parameter. Pengguna dapat
menekan tombol berikut.

- left,right,up,down: putar sudut pandang\\
- +,-: memperbesar atau memperkecil\\
- a: menghasilkan anaglyph (lihat di bawah)\\
- l: ubah mengubah sumber cahaya (lihat di bawah)\\
- space: reset ke default\\
- return: mengakhiri interaksi
\end{eulercomment}
\begin{eulerprompt}
> plot3d("exp(-x^2+y^2)",>user, ...
>   title="Turn with the vector keys (press return to finish)"):
\end{eulerprompt}
\eulerimg{17}{images/EMT4Plot3D_Latifah Nurfitriyani_22305141010-004.png}
\begin{eulercomment}
Jangkauan plot untuk fungsi dapat dispesifikasikan dengan

- a,b: the x-range\\
- c,d: the y-range\\
- r: sebuah persegi simetris sekitar (0,0).\\
- n: jumlah subinterval untuk plot.

Ada beberapa parameter untuk meningkatkan fungsi atau mengubah
tampilan grafik.

fscale: skala ke nilai fungsi (default adalah \textless{}fscale).\\
scale: angka atau 1x2 vektor untuk skala ke arah x dan y.\\
frame: tipe frame (default 1).
\end{eulercomment}
\begin{eulerprompt}
> plot3d("exp(-(x^2+y^2)/5)",r=10,n=80,fscale=4,scale=1.2,frame=3,>user):
\end{eulerprompt}
\eulerimg{17}{images/EMT4Plot3D_Latifah Nurfitriyani_22305141010-005.png}
\begin{eulercomment}
Pandangan dapat diubah dalam berbagai cara.

- distance: jarak pandang ke plot.\\
- zoom: nilai zoom.\\
- angle: sudut ke sumbu y negatif dalam radian.\\
- height: tinggi tampilan dalam radian.

Nilai default dapat diinspeksi atau diubah dengan fungsi view(). Ini
mengembalikan parameter sesuai urutan di atas.
\end{eulercomment}
\begin{eulerprompt}
> view
\end{eulerprompt}
\begin{euleroutput}
  [5,  2.6,  2,  0.4]
\end{euleroutput}
\begin{eulercomment}
Jarak yang lebih dekat membutuhkan sedikit zoom. Efeknya lebih seperti
lensa sudut lebar.

Dalam contoh berikut, angle=0 dan height=0 terlihat dari sumbu y
negatif. Label sumbu y tersembunyi dalam kasus ini.
\end{eulercomment}
\begin{eulerprompt}
> plot3d("x^2+y",distance=3,zoom=2,angle=pi/2,height=0):
\end{eulerprompt}
\eulerimg{17}{images/EMT4Plot3D_Latifah Nurfitriyani_22305141010-006.png}
\begin{eulercomment}
Plot selalu tampak ke tengah dari kubus plot. Anda dapat memindahkan
pusat dengan parameter pusat.
\end{eulercomment}
\begin{eulerprompt}
> plot3d("x^4+y^2",a=0,b=1,c=-1,d=1,angle=-20°,height=20°, ...
>   center=[0.4,0,0],zoom=5.5):
\end{eulerprompt}
\eulerimg{17}{images/EMT4Plot3D_Latifah Nurfitriyani_22305141010-007.png}
\begin{eulercomment}
Plot ini dikecilkan agar muat dalam satuan kubus untuk dilihat. Jadi
tidak perlu mengubah jarak atau zoom tergantung pada ukuran plot.
Namun, label mengacu pada ukuran sebenarnya.

Jika Anda mematikan ini dengan scale=false, Anda perlu berhati-hati,
bahwa plot masih cocok ke jendela plot, dengan mengubah jarak pandang
atau zoom, dan memindahkan pusat.
\end{eulercomment}
\begin{eulerprompt}
> plot3d("5*exp(-x^2-y^2)",r=2,<fscale,<scale,distance=13,height=50°, ...
>  center=[0,0,-2],frame=3):
\end{eulerprompt}
\eulerimg{17}{images/EMT4Plot3D_Latifah Nurfitriyani_22305141010-008.png}
\begin{eulercomment}
Plot kutub juga tersedia. Parameter polar=true menggambar plot polar.
Fungsi harus tetap merupakan fungsi dari x dan y. Parameter "fscale"
skala fungsi dengan skala sendiri. Jika tidak, fungsi ini diperbesar
agar muat dalam kubus.
\end{eulercomment}
\begin{eulerprompt}
> plot3d("1/(x^2+y^2+1)",r=5,>polar, ...
> fscale=2,>hue,n=100,zoom=4,>contour,color=blue):
\end{eulerprompt}
\eulerimg{17}{images/EMT4Plot3D_Latifah Nurfitriyani_22305141010-009.png}
\begin{eulerprompt}
> function f(r) := exp(-r/2)*cos(r); ...
> plot3d("f(x^2+y^2)",>polar,scale=[1,1,0.4],r=pi,frame=3,zoom=5):
\end{eulerprompt}
\eulerimg{17}{images/EMT4Plot3D_Latifah Nurfitriyani_22305141010-010.png}
\begin{eulercomment}
Rotasi parameter memutar fungsi dalam x mengelilingi sumbu x.

- rotate=1: Menggunakan sumbu-x\\
- rotate=2: Menggunakan sumbu-z
\end{eulercomment}
\begin{eulerprompt}
> plot3d("x^2+1",a=-1,b=1,rotate=true,grid=5):
\end{eulerprompt}
\eulerimg{17}{images/EMT4Plot3D_Latifah Nurfitriyani_22305141010-011.png}
\begin{eulerprompt}
> plot3d("x^2+1",a=-1,b=1,rotate=2,grid=5):
\end{eulerprompt}
\eulerimg{17}{images/EMT4Plot3D_Latifah Nurfitriyani_22305141010-012.png}
\begin{eulerprompt}
> plot3d("sqrt(25-x^2)",a=0,b=5,rotate=1):
\end{eulerprompt}
\eulerimg{17}{images/EMT4Plot3D_Latifah Nurfitriyani_22305141010-013.png}
\begin{eulerprompt}
> plot3d("x*sin(x)",a=0,b=6pi,rotate=2):
\end{eulerprompt}
\eulerimg{17}{images/EMT4Plot3D_Latifah Nurfitriyani_22305141010-014.png}
\begin{eulercomment}
Berikut adalah plot dengan tiga fungsi.
\end{eulercomment}
\begin{eulerprompt}
> plot3d("x","x^2+y^2","y",r=2,zoom=3.5,frame=3):
\end{eulerprompt}
\eulerimg{17}{images/EMT4Plot3D_Latifah Nurfitriyani_22305141010-015.png}
\eulerheading{Plot Kontur}
\begin{eulercomment}
Untuk plot, Euler menambahkan garis kisi. Sebaliknya adalah mungkin
untuk menggunakan garis tingkat dan warna satu warna atau warna
spektral berwarna. Euler dapat menggambar ketinggian fungsi pada plot
dengan bayangan. Dalam semua plot 3D Euler dapat menghasilkan anaglif
red/cyan.

- \textgreater{}hue: Menyalakan shading cahaya bukan wires.\\
- \textgreater{}contour: Plot garis kontur otomatis pada plot.\\
- level=... (or levels): Sebuah vektor nilai untuk garis kontur.

Standarnya adalah level="auto", yang menghitung beberapa baris level
secara otomatis. Seperti yang Anda lihat dalam plot, tingkat
sebenarnya adalah kisaran tingkat.

Gaya bawaan dapat diubah. Untuk plot kontur berikut, kami menggunakan
grid yang lebih halus untuk 100x100 poin, skala fungsi dan plot, dan
menggunakan sudut pandang yang berbeda.
\end{eulercomment}
\begin{eulerprompt}
> plot3d("exp(-x^2-y^2)",r=2,n=100,level=1, ...
> >contour,>spectral,fscale=1,scale=1.1,angle=45°,height=20°):
\end{eulerprompt}
\eulerimg{17}{images/EMT4Plot3D_Latifah Nurfitriyani_22305141010-016.png}
\begin{eulerprompt}
> plot3d("exp(x*y)",angle=100°,>contour,color=green):
\end{eulerprompt}
\eulerimg{17}{images/EMT4Plot3D_Latifah Nurfitriyani_22305141010-017.png}
\begin{eulercomment}
Bayangan default menggunakan warna abu-abu. Tapi rentang spektral
warna juga tersedia.

- \textgreater{}spectral: Menggunakan skema spectral default\\
- color=...: Menggunakan warna khusus atau skema spectral

Untuk plot berikut, kami menggunakan skema spectral default dan
meningkatkan jumlah poin untuk mendapatkan tampilan yang sangat halus.
\end{eulercomment}
\begin{eulerprompt}
> plot3d("x^2+y^2",>spectral,>contour,n=100):
\end{eulerprompt}
\eulerimg{17}{images/EMT4Plot3D_Latifah Nurfitriyani_22305141010-018.png}
\begin{eulercomment}
Alih-alih baris level otomatis, kita juga dapat mengatur nilai dari
baris level. Ini akan menghasilkan garis level tipis bukan rentang
levels.
\end{eulercomment}
\begin{eulerprompt}
> plot3d("x^2-y^2",0,5,0,5,level=-1:0.1:1,color=green):
\end{eulerprompt}
\eulerimg{17}{images/EMT4Plot3D_Latifah Nurfitriyani_22305141010-019.png}
\begin{eulercomment}
Dalam plot berikut, kami menggunakan dua bands tingkat yang sangat
luas dari -0.1 ke 1, dan dari 0.9 ke 1. Ini dimasukkan sebagai matriks
dengan batas level sebagai kolom.

Selain itu, kita melapisi grid dengan 10 interval di setiap arah.
\end{eulercomment}
\begin{eulerprompt}
> plot3d("x^2+y^3",level=[-0.1,0.9;0,1], ...
>   >spectral,angle=30°,grid=10,contourcolor=gray):
\end{eulerprompt}
\eulerimg{17}{images/EMT4Plot3D_Latifah Nurfitriyani_22305141010-020.png}
\begin{eulercomment}
Dalam contoh berikut, kita plot set, di mana

\end{eulercomment}
\begin{eulerformula}
\[
f(x,y) = x^y-y^x = 0
\]
\end{eulerformula}
\begin{eulercomment}
Kami menggunakan satu garis tipis untuk garis tingkat.
\end{eulercomment}
\begin{eulerprompt}
> plot3d("x^y-y^x",level=0,a=0,b=6,c=0,d=6,contourcolor=red,n=100):
\end{eulerprompt}
\eulerimg{17}{images/EMT4Plot3D_Latifah Nurfitriyani_22305141010-021.png}
\begin{eulercomment}
Adalah mungkin untuk menunjukkan bidang kontur di bawah plot. Warna
dan jarak ke plot dapat ditentukan.
\end{eulercomment}
\begin{eulerprompt}
> plot3d("x^2+y^4",>cp,cpcolor=green,cpdelta=0.2):
\end{eulerprompt}
\eulerimg{17}{images/EMT4Plot3D_Latifah Nurfitriyani_22305141010-022.png}
\begin{eulercomment}
Berikut adalah beberapa gaya lagi. Kami selalu mematikan frame, dan
menggunakan berbagai skema warna untuk plot dan grid.
\end{eulercomment}
\begin{eulerprompt}
> figure(2,2); ...
>   expr="y^3-x^2"; ...
> figure(1);  ...
>   plot3d(expr,<frame,>cp,>spectral); ...
> figure(2);  ...
>   plot3d(expr,<frame,>spectral,grid=10,cp=2); ...
> figure(3);  ...
>   plot3d(expr,<frame,>contour,color=gray,nc=5,cp=3,colors=red); ...
> figure(4);  ...
>   plot3d(expr,<frame,>hue,grid=10,>transparent,>cp,cpcolor=gray); ...
> figure(0):
\end{eulerprompt}
\eulerimg{17}{images/EMT4Plot3D_Latifah Nurfitriyani_22305141010-023.png}
\begin{eulercomment}
Ada beberapa skema spectral lainnya, bernomor dari 1 sampai 9. Tapi
Anda juga dapat menggunakan color=value, di mana nilai

- spectral: untuk rentang dari biru ke merah\\
- white: untuk rentang yang lebih tipis\\
- yellowblue,purplegreen,blueyellow,greenred\\
- blueyellow, greenpurple,yellowblue,redgreen
\end{eulercomment}
\begin{eulerprompt}
> figure(3,3); ...
> for i=1:9;  ...
>    figure(i); plot3d("x^2+y^2",spectral=i,>contour,>cp,<frame,zoom=4);  ...
> end; ...
> figure(0):
\end{eulerprompt}
\begin{eulercomment}
Sumber cahaya dapat diubah dengan kunci tanah dan kursor selama
interaksi pengguna. Hal ini juga dapat diatur dengan parameter.

- light: arah cahaya\\
- amb: cahaya ambient antara 0 dan 1

Perhatikan bahwa program tidak membuat perbedaan antara sisi plot.
Tidak ada bayangan. Untuk ini Anda akan membutuhkan Povray.
\end{eulercomment}
\begin{eulerprompt}
> plot3d("-x^2-y^2", ...
>   hue=true,light=[0,1,1],amb=0,user=true, ...
>   title="Press l and cursor keys (return to exit)"):
\end{eulerprompt}
\eulerimg{17}{images/EMT4Plot3D_Latifah Nurfitriyani_22305141010-024.png}
\begin{eulercomment}
Parameter warna mengubah warna permukaan. Warna garis tingkat juga
dapat diubah.
\end{eulercomment}
\begin{eulerprompt}
> plot3d("-x^2-y^2",color=rgb(0.2,0.2,0),hue=true,frame=false, ...
>   zoom=3,contourcolor=red,level=-2:0.1:1,dl=0.01):
\end{eulerprompt}
\eulerimg{17}{images/EMT4Plot3D_Latifah Nurfitriyani_22305141010-025.png}
\begin{eulercomment}
Warna 0 memberikan efek pelangi khusus.
\end{eulercomment}
\begin{eulerprompt}
> plot3d("x^2/(x^2+y^2+1)",color=0,hue=true,grid=10):
\end{eulerprompt}
\eulerimg{17}{images/EMT4Plot3D_Latifah Nurfitriyani_22305141010-026.png}
\begin{eulercomment}
Permukaannya juga bisa transparan.
\end{eulercomment}
\begin{eulerprompt}
> plot3d("x^2+y^2",>transparent,grid=10,wirecolor=red):
\end{eulerprompt}
\eulerimg{17}{images/EMT4Plot3D_Latifah Nurfitriyani_22305141010-027.png}
\eulerheading{Plot Implisit}
\begin{eulercomment}
Ada juga plot implisit dalam tiga dimensi. Euler menghasilkan
pemotongan melalui objek. Fitur plot3d termasuk plot implisit. Plot
ini menunjukkan himpunan nol fungsi dalam tiga variabel.\\
Solusi dari

\end{eulercomment}
\begin{eulerformula}
\[
f(x,y,z) = 0
\]
\end{eulerformula}
\begin{eulercomment}
dapat divisualisasikan dalam potongan sejajar dengan x-y-, the x-z-
dan the y-z-plane.

- implisit=1: potong paralel ke y-z-plane\\
- implisit=2: potong paralel ke x-z-plane\\
- implisit=4: potong paralel ke x-y-plane

Tambahkan nilai-nilai ini, jika Anda suka. Dalam contoh kita plot

\end{eulercomment}
\begin{eulerformula}
\[
M = \{ (x,y,z) : x^2+y^3+zy=1 \}
\]
\end{eulerformula}
\begin{eulerprompt}
> plot3d("x^2+y^3+z*y-1",r=5,implicit=3):
\end{eulerprompt}
\eulerimg{17}{images/EMT4Plot3D_Latifah Nurfitriyani_22305141010-028.png}
\begin{eulerprompt}
> c=1; d=1;
> plot3d("((x^2+y^2-c^2)^2+(z^2-1)^2)*((y^2+z^2-c^2)^2+(x^2-1)^2)*((z^2+x^2-c^2)^2+(y^2-1)^2)-d",r=2,<frame,>implicit,>user): 
\end{eulerprompt}
\eulerimg{17}{images/EMT4Plot3D_Latifah Nurfitriyani_22305141010-029.png}
\begin{eulerprompt}
> plot3d("x^2+y^2+4*x*z+z^3",>implicit,r=2,zoom=2.5):
\end{eulerprompt}
\eulerimg{17}{images/EMT4Plot3D_Latifah Nurfitriyani_22305141010-030.png}
\eulerheading{Merencanakan Data 3D}
\begin{eulercomment}
Sama seperti plot2d, plot3d menerima data. Untuk objek 3D, Anda perlu
menyediakan matriks nilai x-, y- dan z, atau tiga fungsi atau ekspresi
fx(x,y), fy(x,y), fz(x,y).

\end{eulercomment}
\begin{eulerformula}
\[
\gamma(t,s) = (x(t,s),y(t,s),z(t,s)))
\]
\end{eulerformula}
\begin{eulercomment}
Karena x, y, z adalah matriks, kita mengasumsikan bahwa (t, s)
berjalan melalui grid persegi. Hasilnya, Anda dapat merencanakan
gambar persegi panjang di ruang angkasa.

Anda dapat menggunakan bahasa matriks Euler untuk menghasilkan
koordinat secara efektif.

Dalam contoh berikut, kita menggunakan vektor nilai t dan vektor kolom
nilai s untuk parameter permukaan bola. Dalam gambar kita dapat
menandai daerah, dalam kasus kita daerah kutub.
\end{eulercomment}
\begin{eulerprompt}
> t=linspace(0,2pi,180); s=linspace(-pi/2,pi/2,90)'; ...
> x=cos(s)*cos(t); y=cos(s)*sin(t); z=sin(s); ...
> plot3d(x,y,z,>hue, ...
> color=blue,<frame,grid=[10,20], ...
> values=s,contourcolor=red,level=[90°-24°;90°-22°], ...
> scale=1.4,height=50°):
\end{eulerprompt}
\eulerimg{17}{images/EMT4Plot3D_Latifah Nurfitriyani_22305141010-031.png}
\begin{eulercomment}
Berikut adalah contoh, yang merupakan grafik fungsi.
\end{eulercomment}
\begin{eulerprompt}
> t=-1:0.1:1; s=(-1:0.1:1)'; plot3d(t,s,t*s,grid=10):
\end{eulerprompt}
\eulerimg{17}{images/EMT4Plot3D_Latifah Nurfitriyani_22305141010-032.png}
\begin{eulercomment}
Namun, kita dapat membuat segala macam permukaan. Berikut adalah
permukaan yang sama sebagai fungsi

\end{eulercomment}
\begin{eulerformula}
\[
x = y \, z
\]
\end{eulerformula}
\begin{eulerprompt}
> plot3d(t*s,t,s,angle=180°,grid=10):
\end{eulerprompt}
\eulerimg{17}{images/EMT4Plot3D_Latifah Nurfitriyani_22305141010-033.png}
\begin{eulercomment}
Dengan lebih banyak upaya, kita dapat menghasilkan banyak permukaan.

Dalam contoh berikut kita membuat tampilan berbayang dari bola
terdistorsi. Koordinat yang biasa untuk bola adalah

\end{eulercomment}
\begin{eulerformula}
\[
\gamma(t,s) = (\cos(t)\cos(s),\sin(t)\sin(s),\cos(s))
\]
\end{eulerformula}
\begin{eulercomment}
dengan

\end{eulercomment}
\begin{eulerformula}
\[
0 \le t \le 2\pi, \quad \frac{-\pi}{2} \le s \le \frac{\pi}{2}.
\]
\end{eulerformula}
\begin{eulercomment}
Kami distorsi ini dengan faktor

\end{eulercomment}
\begin{eulerformula}
\[
d(t,s) = \frac{\cos(4t)+\cos(8s)}{4}.
\]
\end{eulerformula}
\begin{eulerprompt}
> t=linspace(0,2pi,320); s=linspace(-pi/2,pi/2,160)'; ...
> d=1+0.2*(cos(4*t)+cos(8*s)); ...
> plot3d(cos(t)*cos(s)*d,sin(t)*cos(s)*d,sin(s)*d,hue=1, ...
>   light=[1,0,1],frame=0,zoom=5):
\end{eulerprompt}
\eulerimg{17}{images/EMT4Plot3D_Latifah Nurfitriyani_22305141010-034.png}
\begin{eulercomment}
Tentu saja, titik awan juga mungkin. Untuk plot data titik di ruang,
kita perlu tiga vektor untuk koordinat titik.

Gaya-gaya seperti dalam plot 2d dengan points=true;
\end{eulercomment}
\begin{eulerprompt}
> n=500;  ...
>   plot3d(normal(1,n),normal(1,n),normal(1,n),points=true,style="."):
\end{eulerprompt}
\eulerimg{17}{images/EMT4Plot3D_Latifah Nurfitriyani_22305141010-035.png}
\begin{eulercomment}
Hal ini juga dimungkinkan untuk plot kurva dalam 3D. Dalam hal ini,
lebih mudah untuk menghitung titik kurva. Untuk kurva di bidang kami
menggunakan urutan koordinat dan parameter wire=true.
\end{eulercomment}
\begin{eulerprompt}
> t=linspace(0,8pi,500); ...
> plot3d(sin(t),cos(t),t/10,>wire,zoom=3):
\end{eulerprompt}
\eulerimg{17}{images/EMT4Plot3D_Latifah Nurfitriyani_22305141010-036.png}
\begin{eulerprompt}
> t=linspace(0,4pi,1000); plot3d(cos(t),sin(t),t/2pi,>wire, ...
> linewidth=3,wirecolor=blue):
\end{eulerprompt}
\eulerimg{17}{images/EMT4Plot3D_Latifah Nurfitriyani_22305141010-037.png}
\begin{eulerprompt}
> X=cumsum(normal(3,100)); ...
>  plot3d(X[1],X[2],X[3],>anaglyph,>wire):
\end{eulerprompt}
\eulerimg{17}{images/EMT4Plot3D_Latifah Nurfitriyani_22305141010-038.png}
\begin{eulercomment}
EMT juga dapat merencanakan dalam mode aglyph. Untuk melihat plot
seperti itu, Anda perlu kacamata red/cyan glasses.
\end{eulercomment}
\begin{eulerprompt}
> plot3d("x^2+y^3",>anaglyph,>contour,angle=30°):
\end{eulerprompt}
\eulerimg{17}{images/EMT4Plot3D_Latifah Nurfitriyani_22305141010-039.png}
\begin{eulercomment}
Sering kali, skema warna spektral digunakan untuk plot. Hal ini
menekankan ketinggian fungsi.
\end{eulercomment}
\begin{eulerprompt}
> plot3d("x^2*y^3-y",>spectral,>contour,zoom=3.2):
\end{eulerprompt}
\eulerimg{17}{images/EMT4Plot3D_Latifah Nurfitriyani_22305141010-040.png}
\begin{eulercomment}
Euler dapat merencanakan permukaan parameter juga, ketika parameter
adalah x-, y-, dan z-values dari gambar grid persegi panjang di ruang.

Untuk demo berikut, kami menyiapkan parameter u- dan v-, dan
menghasilkan koordinat ruang dari ini.
\end{eulercomment}
\begin{eulerprompt}
> u=linspace(-1,1,10); v=linspace(0,2*pi,50)'; ...
> X=(3+u*cos(v/2))*cos(v); Y=(3+u*cos(v/2))*sin(v); Z=u*sin(v/2); ...
> plot3d(X,Y,Z,>anaglyph,<frame,>wire,scale=2.3):
\end{eulerprompt}
\eulerimg{17}{images/EMT4Plot3D_Latifah Nurfitriyani_22305141010-041.png}
\begin{eulercomment}
Berikut adalah contoh yang lebih rumit, yang megah dengan red/cyan
glasses.
\end{eulercomment}
\begin{eulerprompt}
> u:=linspace(-pi,pi,160); v:=linspace(-pi,pi,400)';  ...
> x:=(4*(1+.25*sin(3*v))+cos(u))*cos(2*v); ...
> y:=(4*(1+.25*sin(3*v))+cos(u))*sin(2*v); ...
>  z=sin(u)+2*cos(3*v); ...
> plot3d(x,y,z,frame=0,scale=1.5,hue=1,light=[1,0,-1],zoom=2.8,>anaglyph):
\end{eulerprompt}
\eulerimg{17}{images/EMT4Plot3D_Latifah Nurfitriyani_22305141010-042.png}
\eulerheading{Rencana Statistik}
\begin{eulercomment}
Plot bar mungkin juga. Untuk ini, kita harus menyediakan

- x: vektor baris dengan elemen n+1\\
- y: vektor kolom dengan elemen n+1\\
- z: matriks nilai nxn.

z dapat lebih besar, tetapi hanya nilai nxn yang akan digunakan.

Dalam contoh, pertama-tama kita menghitung nilai. Kemudian kita
sesuaikan x dan y, sehingga vektor berpusat pada nilai yang digunakan.
\end{eulercomment}
\begin{eulerprompt}
> x=-1:0.1:1; y=x'; z=x^2+y^2; ...
> xa=(x|1.1)-0.05; ya=(y_1.1)-0.05; ...
> plot3d(xa,ya,z,bar=true):
\end{eulerprompt}
\eulerimg{17}{images/EMT4Plot3D_Latifah Nurfitriyani_22305141010-043.png}
\begin{eulercomment}
Adalah mungkin untuk membagi plot permukaan menjadi dua bagian atau
lebih.
\end{eulercomment}
\begin{eulerprompt}
> x=-1:0.1:1; y=x'; z=x+y; d=zeros(size(x)); ...
> plot3d(x,y,z,disconnect=2:2:20):
\end{eulerprompt}
\eulerimg{17}{images/EMT4Plot3D_Latifah Nurfitriyani_22305141010-044.png}
\begin{eulercomment}
Jika memuat atau menghasilkan matriks data M dari berkas dan perlu
plot dalam 3D Anda dapat meningkatkan matriks ke [-1,1] dengan skala
(M), atau meningkatkan matriks dengan \textgreater{}zscale. Hal ini dapat
dikombinasikan dengan faktor skala individu yang diterapkan secara
tambahan.
\end{eulercomment}
\begin{eulerprompt}
> i=1:20; j=i'; ...
> plot3d(i*j^2+100*normal(20,20),>zscale,scale=[1,1,1.5],angle=-40°,zoom=1.8):
\end{eulerprompt}
\eulerimg{17}{images/EMT4Plot3D_Latifah Nurfitriyani_22305141010-045.png}
\begin{eulerprompt}
> Z=intrandom(5,100,6); v=zeros(5,6); ...
> loop 1 to 5; v[#]=getmultiplicities(1:6,Z[#]); end; ...
> columnsplot3d(v',scols=1:5,ccols=[1:5]):
\end{eulerprompt}
\eulerimg{17}{images/EMT4Plot3D_Latifah Nurfitriyani_22305141010-046.png}
\eulerheading{Permukaan Benda Putar}
\begin{eulerprompt}
> plot2d("(x^2+y^2-1)^3-x^2*y^3",r=1.3, ...
> style="#",color=red,<outline, ...
> level=[-2;0],n=100):
\end{eulerprompt}
\eulerimg{17}{images/EMT4Plot3D_Latifah Nurfitriyani_22305141010-047.png}
\begin{eulerprompt}
> ekspresi &= (x^2+y^2-1)^3-x^2*y^3; $ekspresi
\end{eulerprompt}
\begin{eulerformula}
\[
\left(y^2+x^2-1\right)^3-x^2\,y^3
\]
\end{eulerformula}
\begin{eulercomment}
Kami ingin mengubah kurva jantung di sekitar sumbu y. Berikut adalah
ungkapan, yang mendefinisikan hati:

\end{eulercomment}
\begin{eulerformula}
\[
f(x,y)=(x^2+y^2-1)^3-x^2.y^3.
\]
\end{eulerformula}
\begin{eulercomment}
Selanjutnya kita atur

\end{eulercomment}
\begin{eulerformula}
\[
x=r.cos(a),\quad y=r.sin(a).
\]
\end{eulerformula}
\begin{eulerprompt}
> function fr(r,a) &= ekspresi with [x=r*cos(a),y=r*sin(a)] | trigreduce; $fr(r,a)
\end{eulerprompt}
\begin{eulerformula}
\[
\left(r^2-1\right)^3+\frac{\left(\sin \left(5\,a\right)-\sin \left(  3\,a\right)-2\,\sin a\right)\,r^5}{16}
\]
\end{eulerformula}
\begin{eulercomment}
Ini memungkinkan untuk mendefinisikan fungsi numerik, yang
menyelesaikan untuk r, jika a diberikan. Dengan fungsi itu kita dapat
plot jantung berbalik sebagai permukaan parameter.
\end{eulercomment}
\begin{eulerprompt}
> function map f(a) := bisect("fr",0,2;a); ...
> t=linspace(-pi/2,pi/2,100); r=f(t);  ...
> s=linspace(pi,2pi,100)'; ...
> plot3d(r*cos(t)*sin(s),r*cos(t)*cos(s),r*sin(t), ...
> >hue,<frame,color=red,zoom=4,amb=0,max=0.7,grid=12,height=50°):
\end{eulerprompt}
\eulerimg{17}{images/EMT4Plot3D_Latifah Nurfitriyani_22305141010-050.png}
\begin{eulercomment}
Berikut ini adalah plot 3D dari gambar di atas diputar mengelilingi
sumbu-z. Kami mendefinisikan fungsi, yang menggambarkan objek.
\end{eulercomment}
\begin{eulerprompt}
> function f(x,y,z) ...
\end{eulerprompt}
\begin{eulerudf}
  r=x^2+y^2;
  return (r+z^2-1)^3-r*z^3;
   endfunction
\end{eulerudf}
\begin{eulerprompt}
> plot3d("f(x,y,z)", ...
> xmin=0,xmax=1.2,ymin=-1.2,ymax=1.2,zmin=-1.2,zmax=1.4, ...
> implicit=1,angle=-30°,zoom=2.5,n=[10,100,60],>anaglyph):
\end{eulerprompt}
\eulerimg{17}{images/EMT4Plot3D_Latifah Nurfitriyani_22305141010-051.png}
\eulerheading{Plot 3D Khusus}
\begin{eulercomment}
Fungsi plot3d bagus untuk dimiliki, tetapi tidak memenuhi semua
kebutuhan. Selain rutinitas yang lebih mendasar, ada kemungkinan untuk
mendapatkan plot berbingkai dari objek apa pun yang Anda suka.

Meskipun Euler bukan program 3D, Euler dapat menggabungkan beberapa
objek dasar. Kami mencoba untuk memvisualisasikan paraboloid dan
tangennya.
\end{eulercomment}
\begin{eulerprompt}
> function myplot ...
\end{eulerprompt}
\begin{eulerudf}
    y=-1:0.01:1; x=(-1:0.01:1)';
    plot3d(x,y,0.2*(x-0.1)/2,<scale,<frame,>hue, ..
      hues=0.5,>contour,color=orange);
    h=holding(1);
    plot3d(x,y,(x^2+y^2)/2,<scale,<frame,>contour,>hue);
    holding(h);
  endfunction
\end{eulerudf}
\begin{eulercomment}
Sekarang framedplot() menyediakan bingkai, dan mengatur tampilan.
\end{eulercomment}
\begin{eulerprompt}
> framedplot("myplot",[-1,1,-1,1,0,1],height=0,angle=-30°, ...
>   center=[0,0,-0.7],zoom=4):
\end{eulerprompt}
\eulerimg{17}{images/EMT4Plot3D_Latifah Nurfitriyani_22305141010-052.png}
\begin{eulercomment}
Dengan cara yang sama, Anda dapat merencanakan bidang kontur secara
manual. Perhatikan bahwa plot3d() mengatur jendela ke fullwindow()
secara default, tetapi plotcontourplane() mengasumsikan bahwa.
\end{eulercomment}
\begin{eulerprompt}
> x=-1:0.02:1.1; y=x'; z=x^2-y^4;
> function myplot (x,y,z) ...
\end{eulerprompt}
\begin{eulerudf}
    zoom(2);
    wi=fullwindow();
    plotcontourplane(x,y,z,level="auto",<scale);
    plot3d(x,y,z,>hue,<scale,>add,color=white,level="thin");
    window(wi);
    reset();
  endfunction
\end{eulerudf}
\begin{eulerprompt}
> myplot(x,y,z):
\end{eulerprompt}
\eulerimg{27}{images/EMT4Plot3D_Latifah Nurfitriyani_22305141010-053.png}
\eulerheading{Animasi}
\begin{eulercomment}
Euler dapat menggunakan bingkai untuk pra-komputasi animasi.

Salah satu fungsi, yang memanfaatkan teknik ini adalah rotasi. Hal ini
dapat mengubah sudut pandang dan menggambar ulang plot 3D. Fungsi
memanggil addpage() untuk setiap plot baru. Akhirnya animasi plot.

Silakan pelajari sumber rotasi untuk melihat rincian lebih lanjut.
\end{eulercomment}
\begin{eulerprompt}
> function testplot () := plot3d("x^2+y^3"); ...
> rotate("testplot"); testplot():
\end{eulerprompt}
\eulerimg{17}{images/EMT4Plot3D_Latifah Nurfitriyani_22305141010-054.png}
\eulerheading{Menggambar Povray}
\begin{eulercomment}
Dengan bantuan povray.e file Euler, Euler dapat menghasilkan file
Povray. Hasilnya sangat bagus untuk dilihat.

Anda perlu menginstal Povray (32bit atau 64bit) dari
http://www.povray.org/, dan memasukkan sub-direktori "bin" Povray ke jalur lingkungan, atau mengatur variabel "defaultpovray" dengan jalur penuh menunjuk ke "pvengine.exe".

Antarmuka Povray Euler menghasilkan file Povray di direktori rumah
pengguna, dan memanggil Povray untuk mengurai file-file ini. Nama
berkas baku adalah current.pov, dan direktori baku adalah eulerhome(),
biasanya c:\textbackslash{}Users\textbackslash{}Username\textbackslash{}Euler. Povray menghasilkan file PNG, yang
dapat dimuat oleh Euler ke dalam buku catatan. Untuk membersihkan
file-file ini, gunakan povclear().

Fungsi pov3d berada dalam semangat yang sama dengan plot3d. Ini dapat
menghasilkan grafik fungsi f(x,y), atau permukaan dengan koordinat X,
Y, Z dalam matriks, termasuk garis tingkat opsional. Fungsi ini
memulai pelacak sinar secara otomatis, dan memuat adegan ke dalam
notebook Euler.

Selain pov3d(), ada banyak fungsi, yang menghasilkan objek Povray.
Fungsi ini mengembalikan string, berisi kode Povray untuk objek. Untuk
menggunakan fungsi ini, mulai file Povray dengan povstart(). Kemudian
gunakan writeln(...) untuk menulis objek ke berkas adegan. Terakhir,
akhiri file dengan povend(). Secara default, pelacak sinar akan
dimulai, dan PNG akan dimasukkan ke dalam notebook Euler.

Fungsi objek memiliki parameter yang disebut "look", yang membutuhkan
string dengan kode Povray untuk tekstur dan akhir objek. Fungsi
povlook() dapat digunakan untuk menghasilkan string ini. Ini memiliki
parameter untuk warna, transparansi, Phong Shading dll.

Perhatikan bahwa alam semesta Povray memiliki sistem koordinat lain.
Antarmuka ini menerjemahkan semua koordinat ke sistem Povray. Jadi
Anda dapat terus berpikir dalam sistem koordinat Euler dengan z
menunjuk vertikal ke atas, dan sumbu x, y, z dalam arti tangan kanan.\\
Anda perlu memuat file povray.
\end{eulercomment}
\begin{eulerprompt}
> load povray;
\end{eulerprompt}
\begin{eulercomment}
Pastikan, direktori Povray bin berada di path. Jika tidak mengedit
variabel berikut sehingga berisi path ke povray executable.
\end{eulercomment}
\begin{eulerprompt}
> defaultpovray="C:\(\backslash\)Program Files\(\backslash\)POV-Ray\(\backslash\)v3.7\(\backslash\)bin\(\backslash\)pvengine.exe"
\end{eulerprompt}
\begin{euleroutput}
  C:\(\backslash\)Program Files\(\backslash\)POV-Ray\(\backslash\)v3.7\(\backslash\)bin\(\backslash\)pvengine.exe
\end{euleroutput}
\begin{eulercomment}
Untuk kesan pertama, kita merencanakan fungsi sederhana. Perintah
berikut menghasilkan file povray di direktori pengguna Anda, dan
menjalankan Povray untuk melacak file ini.

Jika Anda memulai perintah berikut, Povray GUI harus membuka,
menjalankan file, dan menutup secara otomatis. Karena alasan keamanan,
Anda akan diminta, jika Anda ingin membiarkan file exe berjalan. Anda
dapat menekan cancel untuk menghentikan pertanyaan lebih lanjut. Anda
mungkin harus menekan OK di jendela Povray untuk mengenali dialog
start-up Povray.
\end{eulercomment}
\begin{eulerprompt}
> plot3d("x^2+y^2",zoom=2):
\end{eulerprompt}
\eulerimg{24}{images/EMT4Plot3D_Latifah Nurfitriyani_22305141010-055.png}
\begin{eulerprompt}
> pov3d("x^2+y^2",zoom=3);
\end{eulerprompt}
\eulerimg{27}{images/EMT4Plot3D_Latifah Nurfitriyani_22305141010-056.png}
\begin{eulercomment}
Kita dapat membuat fungsi transparan dan menambahkan penyelesaian
lain. Kita juga dapat menambahkan garis tingkat ke plot fungsi.
\end{eulercomment}
\begin{eulerprompt}
> pov3d("x^2+y^3",axiscolor=red,angle=-45°,>anaglyph, ...
>   look=povlook(cyan,0.2),level=-1:0.5:1,zoom=3.8);
\end{eulerprompt}
\eulerimg{27}{images/EMT4Plot3D_Latifah Nurfitriyani_22305141010-057.png}
\begin{eulercomment}
Kadang-kadang diperlukan untuk mencegah skala fungsi, dan skala fungsi
dengan tangan.

Kami plot set titik dalam bidang kompleks, di mana produk jarak ke 1
dan -1 sama dengan 1.
\end{eulercomment}
\begin{eulerprompt}
> pov3d("((x-1)^2+y^2)*((x+1)^2+y^2)/40",r=2, ...
>   angle=-120°,level=1/40,dlevel=0.005,light=[-1,1,1],height=10°,n=50, ...
>   <fscale,zoom=3.8);
\end{eulerprompt}
\eulerimg{27}{images/EMT4Plot3D_Latifah Nurfitriyani_22305141010-058.png}
\eulerheading{Plotting dengan Koordinat}
\begin{eulercomment}
Alih-alih fungsi, kita dapat plot dengan koordinat. Seperti dalam
plot3d, kita perlu tiga matriks untuk mendefinisikan objek.

Dalam contoh kita mengubah fungsi sekitar sumbu-z.
\end{eulercomment}
\begin{eulerprompt}
> function f(x) := x^3-x+1; ...
> x=-1:0.01:1; t=linspace(0,2pi,50)'; ...
> Z=x; X=cos(t)*f(x); Y=sin(t)*f(x); ...
> pov3d(X,Y,Z,angle=40°,look=povlook(red,0.1),height=50°,axis=0,zoom=4,light=[10,5,15]);
\end{eulerprompt}
\eulerimg{27}{images/EMT4Plot3D_Latifah Nurfitriyani_22305141010-059.png}
\begin{eulercomment}
Dalam contoh berikut, kami merencanakan gelombang yang lembap. Kita
menghasilkan gelombang dengan bahasa matriks Euler.

Kami juga menunjukkan, bagaimana objek tambahan dapat ditambahkan ke
adegan pov3d. Untuk pembuatan objek, lihat contoh berikut. Perhatikan
bahwa plot 3d skala plot, sehingga cocok ke kubus unit.
\end{eulercomment}
\begin{eulerprompt}
> r=linspace(0,1,80); phi=linspace(0,2pi,80)'; ...
> x=r*cos(phi); y=r*sin(phi); z=exp(-5*r)*cos(8*pi*r)/3;  ...
> pov3d(x,y,z,zoom=6,axis=0,height=30°,add=povsphere([0.5,0,0.25],0.15,povlook(red)), ...
>   w=500,h=300);
\end{eulerprompt}
\eulerimg{18}{images/EMT4Plot3D_Latifah Nurfitriyani_22305141010-060.png}
\begin{eulercomment}
Dengan metode bayangan canggih Povray, sangat sedikit titik yang dapat
menghasilkan permukaan yang sangat halus. Hanya pada batas-batas dan
dalam bayangan trik mungkin menjadi jelas.

Untuk ini, kita perlu menambahkan vektor normal di setiap titik
matriks.
\end{eulercomment}
\begin{eulerprompt}
> Z &= x^2*y^3
\end{eulerprompt}
\begin{euleroutput}
  
                                   2  3
                                  x  y
  
\end{euleroutput}
\begin{eulercomment}
Persamaan permukaannya adalah [x,y,Z]. Kami menghitung dua turunan ke
x dan y dari ini dan mengambil produk silang sebagai normal.
\end{eulercomment}
\begin{eulerprompt}
> dx &= diff([x,y,Z],x); dy &= diff([x,y,Z],y);
\end{eulerprompt}
\begin{eulercomment}
Kita mendefinisikan normal sebagai produk silang dari turunan ini, dan
mendefinisikan fungsi koordinat.
\end{eulercomment}
\begin{eulerprompt}
> N &= crossproduct(dx,dy); NX &= N[1]; NY &= N[2]; NZ &= N[3]; N,
\end{eulerprompt}
\begin{euleroutput}
  
                                 3       2  2
                         [- 2 x y , - 3 x  y , 1]
  
\end{euleroutput}
\begin{eulercomment}
Kita hanya menggunakan 25 points.
\end{eulercomment}
\begin{eulerprompt}
> x=-1:0.5:1; y=x';
> pov3d(x,y,Z(x,y),angle=10°, ...
>   xv=NX(x,y),yv=NY(x,y),zv=NZ(x,y),<shadow);
\end{eulerprompt}
\eulerimg{27}{images/EMT4Plot3D_Latifah Nurfitriyani_22305141010-061.png}
\begin{eulercomment}
Berikut ini adalah simpul Trefoil yang dilakukan oleh A. Busser di
Povray. Ada versi yang lebih baik dari ini dalam contoh.

See: Examples\textbackslash{}Trefoil Knot \textbar{} Trefoil Knot

Untuk tampilan yang baik dengan tidak terlalu banyak poin, kami
menambahkan vektor normal di sini. Kami menggunakan Maxima untuk
menghitung normal bagi kita. Pertama, tiga fungsi untuk koordinat
sebagai ekspresi simbolik.
\end{eulercomment}
\begin{eulerprompt}
> X &= ((4+sin(3*y))+cos(x))*cos(2*y); ...
> Y &= ((4+sin(3*y))+cos(x))*sin(2*y); ...
> Z &= sin(x)+2*cos(3*y);
\end{eulerprompt}
\begin{eulercomment}
Kemudian dua vektor turunan ke x dan y.
\end{eulercomment}
\begin{eulerprompt}
> dx &= diff([X,Y,Z],x); dy &= diff([X,Y,Z],y);
\end{eulerprompt}
\begin{eulercomment}
Sekarang normal, yang merupakan produk silang dari dua turunan.
\end{eulercomment}
\begin{eulerprompt}
> dn &= crossproduct(dx,dy);
\end{eulerprompt}
\begin{eulercomment}
Sekarang kita mengevaluasi semua ini secara numerik.
\end{eulercomment}
\begin{eulerprompt}
> x:=linspace(-%pi,%pi,40); y:=linspace(-%pi,%pi,100)';
\end{eulerprompt}
\begin{eulercomment}
Vektor normal adalah evaluasi ekspresi simbolik dn[i] untuk i=1,2,3.
Sintaksis untuk ini adalah \&"expression"(parameters). Ini adalah
alternatif untuk metode dalam contoh sebelumnya, di mana kita
mendefinisikan ekspresi simbolik NX, NY, NZ pertama.
\end{eulercomment}
\begin{eulerprompt}
> pov3d(X(x,y),Y(x,y),Z(x,y),>anaglyph,axis=0,zoom=5,w=450,h=350, ...
>   <shadow,look=povlook(blue), ...
>   xv=&"dn[1]"(x,y), yv=&"dn[2]"(x,y), zv=&"dn[3]"(x,y));
\end{eulerprompt}
\eulerimg{21}{images/EMT4Plot3D_Latifah Nurfitriyani_22305141010-062.png}
\begin{eulercomment}
Kita juga dapat menghasilkan grid dalam 3D.
\end{eulercomment}
\begin{eulerprompt}
> povstart(zoom=4); ...
> x=-1:0.5:1; r=1-(x+1)^2/6; ...
> t=(0°:30°:360°)'; y=r*cos(t); z=r*sin(t); ...
> writeln(povgrid(x,y,z,d=0.02,dballs=0.05)); ...
> povend();
\end{eulerprompt}
\eulerimg{27}{images/EMT4Plot3D_Latifah Nurfitriyani_22305141010-063.png}
\begin{eulercomment}
Dengan povgrid(), kurva dimungkinkan.
\end{eulercomment}
\begin{eulerprompt}
> povstart(center=[0,0,1],zoom=3.6); ...
> t=linspace(0,2,1000); r=exp(-t); ...
> x=cos(2*pi*10*t)*r; y=sin(2*pi*10*t)*r; z=t; ...
> writeln(povgrid(x,y,z,povlook(red))); ...
> writeAxis(0,2,axis=3); ...
> povend();
\end{eulerprompt}
\eulerimg{27}{images/EMT4Plot3D_Latifah Nurfitriyani_22305141010-064.png}
\eulerheading{Objek Povray}
\begin{eulercomment}
Di atas, kami menggunakan pov3d untuk plot permukaan. Antarmuka povray
di Euler juga dapat menghasilkan objek Povray. Objek-objek ini
disimpan sebagai string dalam Euler, dan perlu ditulis ke file Povray.

Kita mulai output dengan povstart().
\end{eulercomment}
\begin{eulerprompt}
> povstart(zoom=4);
\end{eulerprompt}
\begin{eulercomment}
Pertama kita mendefinisikan tiga silinder, dan menyimpannya dalam
string di Euler.

Fungsi povx() dll. cukup mengembalikan vektor [1,0,0], yang dapat
digunakan sebagai gantinya.
\end{eulercomment}
\begin{eulerprompt}
> c1=povcylinder(-povx,povx,1,povlook(red)); ...
> c2=povcylinder(-povy,povy,1,povlook(yellow)); ...
> c3=povcylinder(-povz,povz,1,povlook(blue)); ...
\end{eulerprompt}
\begin{eulercomment}
String mengandung kode Povray, yang kita tidak perlu mengerti pada
saat itu.
\end{eulercomment}
\begin{eulerprompt}
> c2
\end{eulerprompt}
\begin{euleroutput}
  cylinder \{ <0,0,-1>, <0,0,1>, 1
   texture \{ pigment \{ color rgb <0.941176,0.941176,0.392157> \}  \} 
   finish \{ ambient 0.2 \} 
   \}
\end{euleroutput}
\begin{eulercomment}
Seperti yang Anda lihat, kami menambahkan tekstur ke objek dalam tiga
warna yang berbeda.

Itu dilakukan dengan povlook(), yang mengembalikan string dengan kode
Povray yang relevan. Kita dapat menggunakan warna standar Euler, atau
menentukan warna kita sendiri. Kita juga dapat menambahkan
transparansi, atau mengubah cahaya lingkungan.
\end{eulercomment}
\begin{eulerprompt}
> povlook(rgb(0.1,0.2,0.3),0.1,0.5)
\end{eulerprompt}
\begin{euleroutput}
   texture \{ pigment \{ color rgbf <0.101961,0.2,0.301961,0.1> \}  \} 
   finish \{ ambient 0.5 \} 
  
\end{euleroutput}
\begin{eulercomment}
Sekarang kita mendefinisikan objek persimpangan, dan menulis hasilnya
ke file.
\end{eulercomment}
\begin{eulerprompt}
> writeln(povintersection([c1,c2,c3]));
\end{eulerprompt}
\begin{eulercomment}
Persimpangan tiga silinder sulit untuk memvisualisasikan, jika Anda
tidak pernah melihatnya sebelumnya.
\end{eulercomment}
\begin{eulerprompt}
> povend;
\end{eulerprompt}
\eulerimg{27}{images/EMT4Plot3D_Latifah Nurfitriyani_22305141010-065.png}
\begin{eulercomment}
Fungsi berikut menghasilkan fraktal secara rekursif.

Fungsi pertama menunjukkan, bagaimana Euler menangani objek Povray
sederhana. Fungsi povbox() mengembalikan string, berisi koordinat
kotak, tekstur dan akhir.
\end{eulercomment}
\begin{eulerprompt}
> function onebox(x,y,z,d) := povbox([x,y,z],[x+d,y+d,z+d],povlook());
> function fractal (x,y,z,h,n) ...
\end{eulerprompt}
\begin{eulerudf}
   if n==1 then writeln(onebox(x,y,z,h));
   else
     h=h/3;
     fractal(x,y,z,h,n-1);
     fractal(x+2*h,y,z,h,n-1);
     fractal(x,y+2*h,z,h,n-1);
     fractal(x,y,z+2*h,h,n-1);
     fractal(x+2*h,y+2*h,z,h,n-1);
     fractal(x+2*h,y,z+2*h,h,n-1);
     fractal(x,y+2*h,z+2*h,h,n-1);
     fractal(x+2*h,y+2*h,z+2*h,h,n-1);
     fractal(x+h,y+h,z+h,h,n-1);
   endif;
  endfunction
\end{eulerudf}
\begin{eulerprompt}
> povstart(fade=10,<shadow);
> fractal(-1,-1,-1,2,4);
> povend();
\end{eulerprompt}
\eulerimg{27}{images/EMT4Plot3D_Latifah Nurfitriyani_22305141010-066.png}
\begin{eulercomment}
Perbedaan memungkinkan pemotongan satu objek dari objek lain. Seperti
persimpangan, ada bagian dari objek CSG Povray.
\end{eulercomment}
\begin{eulerprompt}
> povstart(light=[5,-5,5],fade=10);
\end{eulerprompt}
\begin{eulercomment}
Untuk demonstrasi ini, kita mendefinisikan objek dalam Povray, bukan
menggunakan string dalam Euler. Definisi ditulis ke file segera.

Koordinat kotak -1 hanya berarti [-1,-1,-1].
\end{eulercomment}
\begin{eulerprompt}
> povdefine("mycube",povbox(-1,1));
\end{eulerprompt}
\begin{eulercomment}
Kita dapat menggunakan objek ini dalam povobject(), yang mengembalikan
string seperti biasa.
\end{eulercomment}
\begin{eulerprompt}
> c1=povobject("mycube",povlook(red));
\end{eulerprompt}
\begin{eulercomment}
Kami menghasilkan kubus kedua, dan memutar dan skala sedikit.
\end{eulercomment}
\begin{eulerprompt}
> c2=povobject("mycube",povlook(yellow),translate=[1,1,1], ...
>  rotate=xrotate(10°)+yrotate(10°), scale=1.2);
\end{eulerprompt}
\begin{eulercomment}
Kemudian kita mengambil perbedaan dari dua objek.
\end{eulercomment}
\begin{eulerprompt}
> writeln(povdifference(c1,c2));
\end{eulerprompt}
\begin{eulercomment}
Sekarang tambahkan tiga sumbu.
\end{eulercomment}
\begin{eulerprompt}
> writeAxis(-1.2,1.2,axis=1); ...
> writeAxis(-1.2,1.2,axis=2); ...
> writeAxis(-1.2,1.2,axis=4); ...
> povend();
\end{eulerprompt}
\eulerimg{27}{images/EMT4Plot3D_Latifah Nurfitriyani_22305141010-067.png}
\eulerheading{Fungsi Implisit}
\begin{eulercomment}
Povray dapat merencanakan himpunan dimana f(x,y,z)=0, sama seperti
parameter implisit dalam plot3d. Namun, hasilnya terlihat jauh lebih
baik.

Sintaksis untuk fungsi sedikit berbeda. Anda tidak dapat menggunakan
keluaran ekspresi Maxima atau Euler.

\end{eulercomment}
\begin{eulerformula}
\[
((x^2+y^2-c^2)^2+(z^2-1)^2)*((y^2+z^2-c^2)^2+(x^2-1)^2)*((z^2+x^2-c^2)^2+(y^2-1)^2)=d
\]
\end{eulerformula}
\begin{eulerprompt}
> povstart(angle=70°,height=50°,zoom=4);
\end{eulerprompt}
\begin{eulercomment}
Buat permukaan implisit. Perhatikan sintaks yang berbeda dalam
ekspresi.
\end{eulercomment}
\begin{eulerprompt}
> writeln(povsurface("pow(x,2)*y-pow(y,3)-pow(z,2)",povlook(green))); ...
> writeAxes(); ...
> povend();
\end{eulerprompt}
\eulerimg{27}{images/EMT4Plot3D_Latifah Nurfitriyani_22305141010-069.png}
\eulerheading{Objek Mesh}
\begin{eulercomment}
Dalam contoh ini, kami menunjukkan bagaimana membuat objek jaring, dan
menggambarnya dengan informasi tambahan.

Kami ingin memaksimalkan xy dalam kondisi x+y=1 dan menunjukkan
sentuhan tangensial dari garis tingkat.
\end{eulercomment}
\begin{eulerprompt}
> povstart(angle=-10°,center=[0.5,0.5,0.5],zoom=7);
\end{eulerprompt}
\begin{eulercomment}
Kita tidak dapat menyimpan objek dalam string seperti sebelumnya,
karena terlalu besar. Jadi kita mendefinisikan objek dalam file Povray
menggunakan #declare. Fungsi povtriangle() melakukan ini secara
otomatis. Ia dapat menerima vektor normal seperti pov3d().

Berikut ini mendefinisikan objek mesh, dan menuliskannya segera ke
dalam file.
\end{eulercomment}
\begin{eulerprompt}
> x=0:0.02:1; y=x'; z=x*y; vx=-y; vy=-x; vz=1;
> mesh=povtriangles(x,y,z,"",vx,vy,vz);
\end{eulerprompt}
\begin{eulercomment}
Sekarang kita mendefinisikan dua cakram, yang akan berpotongan dengan
permukaan.
\end{eulercomment}
\begin{eulerprompt}
> cl=povdisc([0.5,0.5,0],[1,1,0],2); ...
> ll=povdisc([0,0,1/4],[0,0,1],2);
\end{eulerprompt}
\begin{eulercomment}
Tulis permukaan dikurangi dua cakram.
\end{eulercomment}
\begin{eulerprompt}
> writeln(povdifference(mesh,povunion([cl,ll]),povlook(green)));
\end{eulerprompt}
\begin{eulercomment}
Tulis dua persimpangan.
\end{eulercomment}
\begin{eulerprompt}
> writeln(povintersection([mesh,cl],povlook(red))); ...
> writeln(povintersection([mesh,ll],povlook(gray)));
\end{eulerprompt}
\begin{eulercomment}
Tulis titik maksimal.
\end{eulercomment}
\begin{eulerprompt}
> writeln(povpoint([1/2,1/2,1/4],povlook(gray),size=2*defaultpointsize));
\end{eulerprompt}
\begin{eulercomment}
Tambahkan sumbu dan selesaikan.
\end{eulercomment}
\begin{eulerprompt}
> writeAxes(0,1,0,1,0,1,d=0.015); ...
> povend();
\end{eulerprompt}
\eulerimg{27}{images/EMT4Plot3D_Latifah Nurfitriyani_22305141010-070.png}
\eulerheading{Anaglyphs in Povray}
\begin{eulercomment}
Untuk menghasilkan anaglif untuk kacamata merah/sian, Povray harus
berlari dua kali dari posisi kamera yang berbeda. Ini menghasilkan dua
file Povray dan dua file PNG, yang dimuat dengan fungsi
loadanaglyph().

Tentu saja, Anda perlu kacamata merah/sian untuk melihat contoh-contoh
berikut dengan benar.

Fungsi pov3d() memiliki sakelar sederhana untuk menghasilkan sebuah
aglif.
\end{eulercomment}
\begin{eulerprompt}
> pov3d("-exp(-x^2-y^2)/2",r=2,height=45°,>anaglyph, ...
>   center=[0,0,0.5],zoom=3.5);
\end{eulerprompt}
\eulerimg{27}{images/EMT4Plot3D_Latifah Nurfitriyani_22305141010-071.png}
\begin{eulercomment}
Jika Anda membuat adegan dengan objek, Anda perlu menempatkan
pembuatan adegan ke dalam fungsi, dan menjalankannya dua kali dengan
nilai yang berbeda untuk parameter aglif.
\end{eulercomment}
\begin{eulerprompt}
> function myscene ...
\end{eulerprompt}
\begin{eulerudf}
    s=povsphere(povc,1);
    cl=povcylinder(-povz,povz,0.5);
    clx=povobject(cl,rotate=xrotate(90°));
    cly=povobject(cl,rotate=yrotate(90°));
    c=povbox([-1,-1,0],1);
    un=povunion([cl,clx,cly,c]);
    obj=povdifference(s,un,povlook(red));
    writeln(obj);
    writeAxes();
  endfunction
\end{eulerudf}
\begin{eulercomment}
Fungsi povanaglyph() melakukan semua ini. Parameternya seperti di
povstart() dan povend() digabungkan.
\end{eulercomment}
\begin{eulerprompt}
> povanaglyph("myscene",zoom=4.5);
\end{eulerprompt}
\eulerimg{27}{images/EMT4Plot3D_Latifah Nurfitriyani_22305141010-072.png}
\eulerheading{Menentukan Objek sendiri}
\begin{eulercomment}
Antarmuka povray Euler berisi banyak objek. Tapi kau tidak dibatasi
untuk ini. Anda dapat membuat objek sendiri, yang menggabungkan objek
lain, atau objek yang sama sekali baru.

Kami menunjukkan torus. Perintah Povray untuk ini adalah "torus". Jadi
kita kembali string dengan perintah ini dan parameternya. Perhatikan
bahwa torus selalu berpusat pada asal usulnya.
\end{eulercomment}
\begin{eulerprompt}
> function povdonat (r1,r2,look="") ...
\end{eulerprompt}
\begin{eulerudf}
    return "torus \{"+r1+","+r2+look+"\}";
  endfunction
\end{eulerudf}
\begin{eulercomment}
Berikut adalah torus pertama kita
\end{eulercomment}
\begin{eulerprompt}
> t1=povdonat(0.8,0.2)
\end{eulerprompt}
\begin{euleroutput}
  torus \{0.8,0.2\}
\end{euleroutput}
\begin{eulercomment}
Mari kita gunakan objek ini untuk membuat torus kedua, diterjemahkan
dan diputar.
\end{eulercomment}
\begin{eulerprompt}
> t2=povobject(t1,rotate=xrotate(90°),translate=[0.8,0,0])
\end{eulerprompt}
\begin{euleroutput}
  object \{ torus \{0.8,0.2\}
   rotate 90 *x 
   translate <0.8,0,0>
   \}
\end{euleroutput}
\begin{eulercomment}
Sekarang kita tempatkan benda-benda ini ke dalam sebuah adegan. Untuk
tampilan, kami menggunakan Phong Shading.
\end{eulercomment}
\begin{eulerprompt}
> povstart(center=[0.4,0,0],angle=0°,zoom=3.8,aspect=1.5); ...
> writeln(povobject(t1,povlook(green,phong=1))); ...
> writeln(povobject(t2,povlook(green,phong=1))); ...
\end{eulerprompt}
\begin{eulerttcomment}
 >povend();
\end{eulerttcomment}
\begin{eulercomment}
calls the Povray program. However, in case of errors, it does not
display the error. You should therefore use

\end{eulercomment}
\begin{eulerttcomment}
 >povend(<exit);
\end{eulerttcomment}
\begin{eulercomment}

if anything did not work. This will leave the Povray window open.
\end{eulercomment}
\begin{eulerprompt}
>povend(h=320,w=480);
\end{eulerprompt}
\eulerimg{19}{images/EMT4Plot3D_Latifah Nurfitriyani_22305141010-073.png}
\begin{eulercomment}
Berikut adalah contoh yang lebih rumit. Kami memecahkan

\end{eulercomment}
\begin{eulerformula}
\[
Ax \le b, \quad x \ge 0, \quad c.x \to \text{Max.}
\]
\end{eulerformula}
\begin{eulercomment}
dan menunjukkan poin layak dan optimum dalam plot 3D.
\end{eulercomment}
\begin{eulerprompt}
> A=[10,8,4;5,6,8;6,3,2;9,5,6];
> b=[10,10,10,10]';
> c=[1,1,1];
\end{eulerprompt}
\begin{eulercomment}
Pertama, mari kita periksa, apakah contoh ini memiliki solusi sama
sekali.
\end{eulercomment}
\begin{eulerprompt}
> x=simplex(A,b,c,>max,>check)'
\end{eulerprompt}
\begin{euleroutput}
  [0,  1,  0.5]
\end{euleroutput}
\begin{eulercomment}
Ya, benar.

Selanjutnya kita tentukan dua objek. Yang pertama adalah pesawat

\end{eulercomment}
\begin{eulerformula}
\[
a \cdot x \le b
\]
\end{eulerformula}
\begin{eulerprompt}
> function oneplane (a,b,look="") ...
\end{eulerprompt}
\begin{eulerudf}
    return povplane(a,b,look)
  endfunction
\end{eulerudf}
\begin{eulercomment}
Kemudian kita tentukan persimpangan dari semua setengah ruang dan
sebuah kubus.
\end{eulercomment}
\begin{eulerprompt}
> function adm (A, b, r, look="") ...
\end{eulerprompt}
\begin{eulerudf}
    ol=[];
    loop 1 to rows(A); ol=ol|oneplane(A[#],b[#]); end;
    ol=ol|povbox([0,0,0],[r,r,r]);
    return povintersection(ol,look);
  endfunction
\end{eulerudf}
\begin{eulercomment}
Sekarang kita bisa memplot scene.
\end{eulercomment}
\begin{eulerprompt}
> povstart(angle=120°,center=[0.5,0.5,0.5],zoom=3.5); ...
> writeln(adm(A,b,2,povlook(green,0.4))); ...
> writeAxes(0,1.3,0,1.6,0,1.5); ...
\end{eulerprompt}
\begin{eulercomment}
Berikut ini adalah lingkaran di sekitar optimum.
\end{eulercomment}
\begin{eulerprompt}
> writeln(povintersection([povsphere(x,0.5),povplane(c,c.x')], ...
>   povlook(red,0.9)));
\end{eulerprompt}
\begin{eulercomment}
Dan kesalahan dalam arah optimum.
\end{eulercomment}
\begin{eulerprompt}
> writeln(povarrow(x,c*0.5,povlook(red)));
\end{eulerprompt}
\begin{eulercomment}
Kami menambahkan teks ke layar. Teks hanya objek 3D. Kita perlu
menempatkan dan mengubahnya sesuai dengan pandangan kita.
\end{eulercomment}
\begin{eulerprompt}
> writeln(povtext("Linear Problem",[0,0.2,1.3],size=0.05,rotate=5°)); ...
> povend();
\end{eulerprompt}
\eulerimg{27}{images/EMT4Plot3D_Latifah Nurfitriyani_22305141010-076.png}
\eulerheading{Contoh lain}
\begin{eulercomment}
Anda dapat menemukan beberapa contoh lagi untuk Povray di Euler di
file berikut.

See: Examples/Dandelin Spheres\\
See: Examples/Donat Math\\
See: Examples/Trefoil Knot\\
See: Examples/Optimization by Affine Scaling
\end{eulercomment}
\end{eulernotebook}
\end{document}
